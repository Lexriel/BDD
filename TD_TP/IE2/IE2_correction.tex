\documentclass{article}

\usepackage[utf8]{inputenc} %%pour taper directement les lettres accentuées
\usepackage[francais]{babel}
\usepackage[T1]{fontenc}

\usepackage{anysize} %%pour pouvoir mettre les marges qu'on veut
\marginsize{2.5cm}{2.5cm}{1.5cm}{1.5cm}

% \usepackage{indentfirst} %%pour que les premier paragraphes soient aussi indentés
\usepackage{amsfonts,amssymb,amsmath,amsthm}
\usepackage{hyperref}
% \hypersetup{pdfstartview=XYZ}
\usepackage{enumerate}
\usepackage{array}
\usepackage{verbatim}
\usepackage{eurosym}
\usepackage{amssymb}
\newcommand{\et}{\mbox{ and }}
\newcommand{\ou}{\mbox{ or }}

\begin{document}

\begin{center}
{\Large \textbf{IE2 de BDD - correction}}
\end{center}

\noindent\hrulefill
\vspace{0.5cm}

% \vspace{2.5cm}
\noindent
\Large{\textbf{Exercice 1 - TRC}}

\normalsize
\vspace{0.2cm}
\noindent
\textbf{Question 1.1 (1pt)}

\noindent
La deuxième ligne nous indique que nous recherchons des identifiants
de vols $t[vid]$ qui peuvent être volés ($v[distance]\leq a[portee]$)
avec des avions affectés à des pilotes $p$ (jointure sur les $aid$).

\noindent
Les troisième et quatrième lignes nous indiquent que ces vols ont une
autre propriété. Si on considère un vol $v$, tout avion qui pourrait
le voler avec d'autres pilotes $p_2$ implique que $p_2=p$ (avec
$c2[eid]=c[eid]$). En d'autres termes, il n'y a pas d'autres pilotes
que $p$ qui peuvent le voler.

\noindent
On conclue alors que cette requête renvoie les identifiants de vols
qui ne peuvent être volés que par un seul pilote.

\vspace{0.3cm}
\noindent
\textbf{Question 1.2 (1pt)}

\noindent
On va modifier la requête TRC pour faire apparaître une forme que l'on
peut écrire en SQL.

% \noindent
% $
% \{t | \exists v \in vols, \exists a \in avions, \exists c \in certifications:\\
% t[vid]=v[vid] \et a[aid]=c[aid] \et v[distance]\leq a[portee] \et \\
% \forall c2 \in certifications,\forall a2 \in avions : (c2[aid]=a2[aid] \et v[distance]\leq a2[portee]) \\
% \Rightarrow c2[eid] = c[eid] \}
% $

\vspace{0.1cm}
\noindent
On supprime l'implication à l'aide de la formule de De Morgan :

\noindent
$
\{t | \exists v \in vols, \exists a \in avions, \exists c \in certifications:\\
t[vid]=v[vid] \et a[aid]=c[aid] \et v[distance]\leq a[portee] \et \\
\forall c2 \in certifications,\forall a2 \in avions : not(c2[aid]=a2[aid] \et v[distance]\leq a2[portee]) \ou \\
c2[eid] = c[eid] \}
$

\vspace{0.1cm}
\noindent
On supprime les "pour tout" :

\noindent
$
\{t | \exists v \in vols, \exists a \in avions, \exists c \in certifications:\\
t[vid]=v[vid] \et a[aid]=c[aid] \et v[distance]\leq a[portee] \et \\
\nexists c2 \in certifications,\nexists a2 \in avions : not(not(c2[aid]=a2[aid] \et v[distance]\leq a2[portee]) \ou \\
c2[eid] = c[eid]) \}
$

\vspace{0.1cm}
\noindent
On supprime les doubles négations :

\noindent
$
\{t | \exists v \in vols, \exists a \in avions, \exists c \in certifications:\\
t[vid]=v[vid] \et a[aid]=c[aid] \et v[distance]\leq a[portee] \et \\
\nexists c2 \in certifications,\nexists a2 \in avions : c2[aid]=a2[aid] \et v[distance]\leq a2[portee]) \et \\
c2[eid] \neq c[eid]) \}
$

\vspace{0.1cm}
\noindent
On peut maintenant écrire la requête en SQL ainsi :
\begin{verbatim}
SELECT vid FROM vols v, avions a NATURAL JOIN certifications c WHERE
v.distance<=a.portee and not EXISTS(
  SELECT * FROM avions a2 NATURAL JOIN certifications c2 WHERE
  a2.portee>=v.distance and c2.eid<>c.eid
);
\end{verbatim}

\vspace{0.3cm}
\noindent
\Large{\textbf{Exercice 2 - Normalisation}}

\normalsize
\vspace{0.2cm}
\noindent
Dans tout cet exercice, les attributs formant la clé d'une relation
sont soulignés.

\vspace{0.2cm}
\noindent
\textbf{Question 2.1 (1pt)}

\noindent
On considère $R(\underline{MedecinID}, \underline{PatientID}, \underline{Date}, Diagnose, CodeTraitement, Tarif)$.

\noindent
On définit les dépendances fonctionnelles par

\noindent
$F = \{\ [MedecinID, PatientID,Date] \rightarrow [CodeTraitement, Diagnose],
[CodeTraitement] \rightarrow [Tarif]\ \}$.

\vspace{0.3cm}
\noindent
\textbf{Question 2.2 (1pt)}

\noindent
La relation $R$ est en 2NF car il n'y a pas de dépendances partielles.

\vspace{0.3cm}
\noindent
\textbf{Question 2.3 (1pt)}

\noindent
La relation $R$ n'est pas en 3NF car l'attribut non-primaire $Tarif$ est dépendant transitivement de la clé :

\noindent
$[MedecinID, PatientID, Date] \rightarrow [CodeTraitement] \rightarrow [Tarif]$.

\noindent
On décompose $R$ en $(R_1,R_2)$ qui est 3NF avec

\noindent
$R_{1} =
(\underline{MedecinID}, \underline{PatientID}, \underline{Date}, Diagnose, CodeTraitement)$ et

\noindent
$R_{2} =
(\underline{CodeTraitement}, Tarif)$.

\end{document}
